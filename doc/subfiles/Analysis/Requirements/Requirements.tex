\documentclass[../../../main.tex]{subfiles}
\begin{document}
\chapter{Features}
From my research and analysis I have created a set of features that I aim to implement. I have split these requirements into development iterations that they will be linked with. Each iteration has a theme that most of the requirements will follow.
\begin{itemize}
	\item Iteration I - Emphasis on Core Functionality
		\begin{itemize}
			\item Plot a explicit function in $x$
			\item Plot multiple functions on the same plot
		\end{itemize}
	\item Iteration II - Emphasis on the User
		\begin{itemize}
			\item Let the user input functions
			\item Zoom in/out of the graph
			\item Pan around the graph
			\item Plot special functions such as trigonometric functions, logarithms, modulus functions, etc.
			\item Identify intersection with the axes
			\item Multiple plots that you can switch between
			\item Save plots as pictures
			\item Save workspace to resume later
			\item Dark Theme
		\end{itemize}
	\item Iteration III - Advanced Features
		\begin{itemize}
			\item Identify intersections between functions
			\item Differentiate explicit continuous functions with respect to $x$
			\item Polar equations
			\item Implicit equations
			\item Parametric equations
			\item \LaTeX \ equation support
			\item Identify turning points 
		\end{itemize}
\end{itemize}

\chapter{Success Criteria}
Here is a list of success criteria that I will evaluate my prototypes against:
\begin{table}[H]
\centering
\begin{tabular}{|C{0.3\textwidth}|C{0.6\textwidth}|}
\hline
Criteria                                                                        & Justification                                                                                                                                                                    \\ \hline
Plots Polynomials, Exponentials and Rational Functions.                         & These are the functions that an A-Level Student would have to know how to sketch and as such my program should be able to sketch.                                                \\ \hline
The accuracy of plotting function should be comparable to other graphing programs such as Desmos.                & A-Level students will be using this program and they shouldn't be misled about what a function looks like.                                                                       \\ \hline
The control of the viewport should be responsive.                               & One of my stakeholders, Matthew, said that he hated it when a graphing program lags, and so my program should not lag for most functions.                                        \\ \hline
There are no major experience affecting bugs in the software.                   & The experience should be pleasant for the user since bugs are painful for the user.                                                                                                \\ \hline
The software should be able to run on most hardware since at least 7 years ago. & This links to the idea that the program should be responsive but also the fact that it should be accessible to anyone who wishes to use it irrelevant of the hardware they have. \\ \hline
\end{tabular}
\end{table}
I will use \texttt{Java} to accomplish this task. \texttt{Java} is platform independent meaning that I do not have to compile for many different systems. It also means that if I want to create a mobile version I will only have to recreate the UI. Within \texttt{Java} I will use \texttt{JavaFX} for my UI. \texttt{JavaFX} allows for customization of UI elements through \texttt{css} files, as well as an easy interface to draw objects onto through the screen, through its \texttt{Canvas} class.  I am also comfortable with the language which makes it the most suitable language for this project.
\end{document}