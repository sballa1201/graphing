\documentclass[../../../../../main.tex]{subfiles}
\begin{document}

\subsection{Parsing Algorithm}
\subsubsection{Least Significant Operator}
An important part of the parsing stage, is the ability to find the least significant operator within an expression. If there are no operators $-1$ will be returned. If an operator is inbetween within a pair of brackets it is automatically discounted. We do this by respectively incrementing or decrementing a variable when we reach an opening or closing bracket to monitor when we are within a pair of brackets.
\begin{algorithm}
\caption{Least Significant Operator Position}
\DontPrintSemicolon
\Fn{leastSigOperatorPos(\KwString input)}{
	\KwInt parenthesis = 0 \;
	\KwInt leastSigOperatorPos = -1 \tcc{stores the position of the least significant operator so far}
	\KwInt leastSigOpcode = 1000\;
	\KwCharA operators = [``+'',' ``-'', ``/'', `` * '', `` \^ ''] \tcc{stores each operator in order of incresing significance}
	\KwInt currentOpcode \tcc{the current index of the operator in the array operators}
	\KwChar currentChar\;
	\For{i=0 \KwTo (input.size - 1) \KwBy 1}{
		currentChar = input[i]\;
		\uIf {currentChar \KwIn operators} {
			currentOpcode = operators.find(currentChar)\;
			\uIf{(currentOpcode $\leq$ leastSigOpcode) \KwAnd (parenthesis $==$ 0)}{
				leastSigOperatorPos = i\;
				leastSigOpcode = currentOpcode \tcc{Update the least significant operator so far, as it is now the current character}
				
			}
			 
		}\uElseIf{currentChar == ``(''} {
			parenthesis$++$\;		
		}\ElseIf{currentChar == ``)''} {
			parenthesis$--$\;		
		}
	}
	\KwRet leastSigOperatorPos\;
}
\end{algorithm}
\subsubsection{Removing Brackets}
The function above assumes that there is no whitespace and that there are no brackets enclosing the entire expression (e.g.\ $(x-4)$). We can deal with our whitespace issue in our constructor\footnote{In our function class we will store the original input so we can show the user, therefore we do not need to remove whitespace here} however we need to make another function check for and remove any brackets surrounding an expression.

However algorithm \ref{alg:checkBracket1} has some issues. For example if we have the expression $\frac{x+1}{x+2}$, this would be input as, $(x+1)/(x+2)$. Now if we apply algorithm \ref{alg:checkBracket1} which removes enclosing brackets we get $x+1)/(x+2$. This is completely wrong, as in this case we do not want to any remove brackets at all. The significant issue here is that we only want to remove the enclosing brackets, if they are \textit{\textbf{matching}}.

To do this the algorithm \ref{alg:checkBracket2} is more suited. This algorithm is based on \ref{alg:checkBracket1} but checks for matching brackets. It does this by using a variable that increments everytime there is a opening bracket and decrements every time there is a closing bracket. When the variable becomes 0 the matching bracket has been found. If this happens at the end the opening and closing brackets are removed, else nothing happens. 
This algorithm will also throw an exception if there are unequal number of opening and closing brackets. This is so that we can inform the user later of the error that they have made and so that we can kill the process instantly rather than letting this error have consequences later on (probably during the evaluation of a value).
\begin{algorithm}
\DontPrintSemicolon
\caption{Check for and remove any Brackets surrounding an input}
\label{alg:checkBracket1}
\Fn{checkBracket(\KwString input)}{
	\KwBool done = \textit{\textbf{False}} \;
	\While{!done}{
		done = \textit{\textbf{True}} \;
		\If{(input[0] == `(') \KwAnd (input[input.size - 1] == `)')}{
			done = \textit{\textbf{False}} \;
			input = input.substring(1, input.size - 2)\;
		}
	}
	\KwRet input\;
}
\end{algorithm}

\begin{algorithm}[H]
\DontPrintSemicolon
\caption{Check for and remove any Matching Brackets surrounding an input}
\label{alg:checkBracket2}
\Fn{checkBracket(\KwString input)}{
	\KwBool done = \textit{\textbf{False}} \;
	\While{!done}{
		done = \textit{\textbf{True}} \;
		\uIf{input[0] == `(' \KwAnd input[input.size - 1] == `)'}{
			\KwInt countMatching = 1\;
			\For{i=1 \KwTo (input.size - 2) \KwBy 1}{
				\uIf{countMatching == 0}{
					\KwRet input\;
				} \uElseIf{input[i] == `(' }{
					countMatching++
				} \ElseIf{input[i] == `)'} {
					countMatching--\;
				}
			}
			\eIf{countMatching == 1}{
				done = \textit{\textbf{False}}\;
				input = input.substring(1, input.size - 2)\;
			}{
				\KwThrow ``There is an unequal number of opening and closing brackets''\;
			}
		
		} 
		
		
	}
	\KwRet input\;
}
\end{algorithm}
\newpage
\subsubsection{RegEx}
Another important part of the parsing stage is to standardize the input. This is where we convert any inconsistencies discussed in section \ref{funcClass}, page \pageref{funcClass}. The easiest way to do this is to use \texttt{RegEx\cite{regex}}. \texttt{RegEx} stands for regular expression and is a standardized form of pattern recognition in strings, usually used during syntax analysis during compilation of software. Many languages support \texttt{RegEx} in some form or another and Java is no exception. These were our 5 inconsistencies that we needed to fix:
\begin{enumerate}
	\item Any instance of $ax$ where $a \in  \mathbb{R} : a \neq 0$ is to be converted to $a*x$.
	\item Any instance of $a($ and $)a$ where $a$ is not an operator, is to be converted to $a*($ and $)*a$ respectively.
	\item Any instance of $(f(x))(g(x))$ is to be converted to $(f(x))*(g(x))$.
	\item Any instance of $!-f(x)$ where $!$ is to be any operator (e.g. $*$ or $/$) is to be converted to $! (-f(x))$.
	\item Any instance of $-f(x)$ at the start or next to an opening bracket is to be converted to $0 - f(x)$.
\end{enumerate}
For the first and second examples, we look around every instance of $x$, $($ or $)$ and if the adjacent characters are not operators or brackets then we replace with $*x$ or $x*$. Therefore we can combine the first and second examples to use two separate \texttt{RegEx} expressions to deal with the case where we have $x$ after and where we have $x$ before.\par
The \texttt{RegEx} expression for the first case is
\textbf{\underline{``([$\wedge$\textbackslash(\textbackslash)\textbackslash+\textbackslash-\textbackslash*\textbackslash/\textbackslash$\wedge$])([\textbackslash(a-z])''}}
with the replacement expression being \textbf{\underline{``\$1*\$2''}}. If we take the \texttt{RegEx} expression, it creates two capture groups, \textit{``\$1''} and \textit{``\$2''}, which are 
\textit{``([$\wedge$\textbackslash(\textbackslash)\textbackslash+\textbackslash-\textbackslash*\textbackslash/\textbackslash$\wedge$])''}
 and 
 \textit{``([\textbackslash\textbackslash(a-z])''} 
 respectively. A capture group stores a set of characters for each match that is made, so that we can perform actions on it later. The first capture group checks if the first character, in the substring that is currently being checked, is not any of the operators or brackets\footnote{the reason there are so many backslashes is because a lot of the operators are actually key characters in \texttt{RegEx} and a blackslash is an escape character which means that it signifies to treat the next character as a pure character}. The not is signified by the first $\wedge$. The second capture group checks if the second character, in the substring that is currently being checked, is a constant/variable
 \footnote{While we are only dealing with the variable $x$ at the moment, this allows for constants to be used in the future and be dealt with correctly.}
 , signified by the \textit{``a-z''} or is an opening bracket \textit{``(''}. If both capture groups return true then a match is found and the match is replaced with \textit{``\$1*\$2''} where \textit{``\$1''} is the first capture group, \textit{``\$2''} is the second capture group and the * asterisk between them signifying the multiply.
 
The \texttt{RegEx} expression for the second case is
\textbf{\underline{``([\textbackslash)a-z])([$\wedge$\textbackslash(\textbackslash)\textbackslash+\textbackslash-\textbackslash*\textbackslash/\textbackslash$\wedge$])''}}
with the replacement expression being \textbf{\underline{``\$1*\$2''}} again. This expression does the same as the first but checks for the reverse order i.e.\ $xa$ and instead checks for a ) instead of ( as we are checking the back of a substring instead of the start.

For the third inconsistency, the \texttt{RegEx} expression is
\textbf{\underline{``\textbackslash)\textbackslash(''}}
with the replacement expression being \textbf{\underline{``)*(''}}. This expression returns a match if it finds a ) followed by a (. If it finds a match it then replaces the entire match with \textit{``)*(''}.

For the fourth inconsistency, the \texttt{RegEx} expression is
\textbf{\underline{``([\textbackslash+\textbackslash-\textbackslash*\textbackslash/\textbackslash$\wedge$])-([$\wedge$\textbackslash+\textbackslash-\textbackslash*\textbackslash/\textbackslash(\textbackslash)\textbackslash$\wedge$]*)''}}
with the replacement expression being \textbf{\underline{``\$1(-\$2)''}}. The \texttt{RegEx} expression returns a match when there is an operator followed by a minus sign followed by any number of characters that are not operators or brackets. There are two capture groups. The first is \textit{``([\textbackslash+\textbackslash-\textbackslash*\textbackslash/\textbackslash$\wedge$])''} and this captures the operator. The second is \textit{``([$\wedge$\textbackslash+\textbackslash-\textbackslash*\textbackslash/\textbackslash(\textbackslash)\textbackslash$\wedge$]*)''} and this captures the expression after the minus sign. The match is then replaced by the first capture group, followed by an opening bracket, a minus sign, the second capture group, then a closing bracket.

For the fifth inconsistency, the \texttt{RegEx} expression is
\textbf{\underline{``($\wedge$$|$\textbackslash()-''}}
with the replacement expression being \textbf{\underline{``\$10-''}}. The \texttt{RegEx} expression returns a match when it is either the start of a line, signified by the $\wedge$, or\footnote{the or keyword in \texttt{RegEx} is signified by $|$} a ( followed by a minus sign. The start of the line of bracket is captured and is used in the replacement expression, when the match is replaced with a 0- preceded by either a bracket or nothing depending on if the start of a line or an opening bracket was captured.
\newpage
\subsubsection{Creating the Tree and the Stack}
Using the functions above we can create a syntax tree from our expression. Before we dive into this, remember that we will convert this into a \texttt{Stack} later to process, since it is more efficient (section \ref{stackBetter}). To create the \texttt{Tree} we will use the Divide and Conquer methodology by recursively splitting the original expression until it becomes a single constant or variable. We can know if we should split the expression and if so where we should split the function by using our Least Significant Operator function. Our recursive case will return a \texttt{Tree} where the left and right nodes are made up of the trees of the two sub-expressions and the root node will be the operator that we split our original expression with. Our base case will return a \texttt{Tree} containing the constant or variable that is remaining. For example if we have the expression ``$x^2 + 4$'', we will first split this expression by the least significant operator which is the ``$+$''. We then make this our root node and our left and right trees will be what is left and right of that operator. This will be repeated for each sub-tree until only no operators remain. So our bottom nodes will be $x$, 2 and 4.

\begin{algorithm}[H]
\DontPrintSemicolon
\caption{Create a Binary Tree for an Algebraic Expression}
\label{alg:createTree1}
\Fn{createTree(\KwString expression)}{
	expression = checkBracket(expression)\;
		\KwInt leastSigOperatorPos = leastSigOperatorPos(expression)\;
		\eIf{leastSigOperatorPos == -1} {
			\KwRet \KwNew Tree(expression)\;
		}{
			\KwString operator = expression[leastSigOperatorPos]\;
			\KwString a = expression.substring(0, leastSigOperatorPos)\;
			\KwString b = expression.substring(leastSigOperatorPos+1,expression.length - 1)\;
			\KwRet \KwNew \KwTree (operator,createTree(a),createTree(b))\;
		}		
}
\end{algorithm}
Within this algorithm I can use concurrent processing to process each sub-expression individually. To do this I will create 2 threads to create the 2 individual sub-trees. I will then start these threads and on the root thread wait for these threads to finish (usually done using a method called ``join''), and then return the new \texttt{Tree} object made up of the 2 individual sub-tree.

\begin{algorithm}[H]
\DontPrintSemicolon
\caption{Create a Binary Tree for an Algebraic Expression (Using Concurrent Processing)}
\label{alg:createTree2}
\Fn{createTree(\KwString expression)}{
	expression = checkBracket(expression)\;
		\KwInt leastSigOperatorPos = leastSigOperatorPos(expression)\;
		\eIf{leastSigOperatorPos == -1} {
			\KwRet \KwNew Tree(expression)\;
		}{
			\KwString operator = expression[leastSigOperatorPos]\;
			\KwString a = expression.substring(0, leastSigOperatorPos)\;
			\KwString b = expression.substring(leastSigOperatorPos+1,expression.length - 1)\;
			\KwThread threadA = \KwNew \KwThread(createTree(a))\;
			\KwThread threadB = \KwNew \KwThread(createTree(b))\;
			threadA.join()\;
			threadB.join()\;
			\KwRet \KwNew \KwTree (operator,createTree(a),createTree(b))\;
		}
}
\end{algorithm}
\newpage
\end{document}