\documentclass[../../../../main.tex]{subfiles}
\begin{document}

\section{Trees}
Our trees will be simpler to implement than our stacks. This is because we do not need to use generics with our trees since we will only be storing strings within them. The only hard part of this implementation was the fact that you cannot pass by reference\cite{byRefJava} in \texttt{Java}. I got around this issue by simply passing this list in, and returning it each time. Here is the implementation I have created in \texttt{Java}:
\begin{minted}[
frame=lines,
framesep=2mm,
linenos,
breaklines
]{java}
package structures;

import exceptions.StackOverflowException;

public class BinaryTree {

	// attributes
	// children
	public BinaryTree left;
	public BinaryTree right;
	// node value
	private String value;

	// methods
	// constructor - create a null tree
	public BinaryTree() {
		this.left = null;
		this.right = null;
		this.value = null;
	}

	// constructor - create leaf tree i.e. no children
	public BinaryTree(String value) {
		this.value = value;
		this.left = null;
		this.right = null;
	}

	// constructor - create a tree with the children defined
	public BinaryTree(String value, BinaryTree left, BinaryTree right) {
		this.value = value;
		this.left = left;
		this.right = right;
	}

	// return the value of the current node
	public String getValue() {
		return value;
	}

	// set the value of the current node
	public void setValue(String value) {
		this.value = value;
	}

	// count the number of nodes beneath this tree + this tree
	public int countNodes() {
		// set the number of nodes counted to 0
		int count = 0;

		// recursive case, counts the number of nodes in its children
		// if they are not null
		if (this.left != null) {
			count = count + this.left.countNodes();
		}
		if (this.right != null) {
			count = count + this.right.countNodes();
		}
		// return the count of the children
		// but increment it to count it self
		return count + 1;
	}

	// post-order depth-first traversal wrapper function
	public Stack<String> traverse() throws StackOverflowException {
		// create a stack of max height
		// which is the number of nodes in the entire tree
		// this is to save memory
		Stack<String> order = new Stack<String>(this.countNodes());
		// traverse the tree
		order = traverseHelper(this, order);
		return order;
	}

	// the actual recursive traversal function
	private static Stack<String> traverseHelper(BinaryTree tree, Stack<String> order) throws StackOverflowException {
		// base case, if the tree is null just return the unmodified stack
		if (tree == null) {
			return order;
		} else { // recursive case search its children if not null
			// since this is post-order, the order is left, right then root
			// traverse left
			order = traverseHelper(tree.left, order);
			// traverse right
			order = traverseHelper(tree.right, order);
			// finally push the root value into the stack
			order.push(tree.value);
			return order;
		}
	}

}
\end{minted}
\newpage
\noindent
Just to test my binary tree implementation I tried to recreate the tree below:\\
\par
\Tree[.G 
		[.C A B ]
		[.F D E ] 
]
\bigskip \\
If this graph has post-order, depth first traversal performed on it then the output should be:
\[
\begin{bmatrix}
A & B & C & D & E & F & G
\end{bmatrix}
\]
Here is the test code:
\begin{minted}[breaklines]{java}
BinaryTree root;
BinaryTree left = new BinaryTree("C", new BinaryTree("A"), new BinaryTree("B"));
BinaryTree right = new BinaryTree("F", new BinaryTree("D"), new BinaryTree("E"));
root = new BinaryTree("G",left,right);
// print the traversal stack
System.out.println(root.traverse());
\end{minted}
Here is the output:
\begin{minted}[breaklines]{console}
[G], F, E, D, C, B, A
\end{minted}
While this appears to be different to our expected stack it isn't. If we look closely the items are actually reversed, which makes sense since a stack is FILO (First In Last Out) structure and we made the visual representation of our stack implementation reflect this property. This means that \textit{A} was pushed into the stack first, which means our tree implementation is working as intended.
\newpage
\end{document}